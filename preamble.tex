\documentclass[a4paper,12pt]{article} % добавить leqno в [] для нумерации слева

%%% Работа с русским языком
\usepackage{cmap}					% поиск в PDF
\usepackage{mathtext} 				% русские буквы в формулах
\usepackage[T2A]{fontenc}			% кодировка
\usepackage[utf8]{inputenc}			% кодировка исходного текста

% last lang has priority
\usepackage[russian, english]{babel}	% локализация и переносы

%%% Дополнительная работа с математикой
\usepackage{amsmath,amsfonts,amssymb,amsthm,mathtools} % AMS
\usepackage{icomma} % "Умная" запятая: $0,2$ --- число, $0, 2$ --- перечисление

%% Номера формул
\mathtoolsset{showonlyrefs=true} % Показывать номера только у тех формул, на которые есть \eqref{} в тексте.

%% Шрифты
\usepackage{euscript}	 % Шрифт
\usepackage{mathrsfs} % Красивый матшрифт

%% Свои команды
\DeclareMathOperator{\sgn}{\mathop{sgn}}
\newcommand{\eqdef}{\stackrel{\rm def}{=}} 

%% Перенос знаков в формулах (по Львовскому)
\newcommand*{\hm}[1]{#1\nobreak\discretionary{}
	{\hbox{$\mathsurround=0pt #1$}}{}}

%%% Заголовок
\author{Alexander Petrov}
\title{Solutions}
\date{\today}


%%% Теоремы
\theoremstyle{plain} % Это стиль по умолчанию, его можно не переопределять.
\newtheorem{theorem}{Th}[section]
\newtheorem{lemma}{Le}[section]
\newtheorem{proposition}{Утверждение}[section]


\theoremstyle{definition} % "Определение"
\newtheorem{problem}{Задача}[section]
\newtheorem{defin}{Def}[section]
\newtheorem{n}{№}
\newtheorem{note}{Note}[section]

\theoremstyle{remark} % "Примечание"
\newtheorem*{ev}{Доказательство}
\newtheorem*{id}{Идея доказательства}

\usepackage{mathrsfs}

%%% Работа с картинками
\usepackage{graphicx}  % Для вставки рисунков
\graphicspath{{images/}}  % папки с картинками
\setlength\fboxsep{3pt} % Отступ рамки \fbox{} от рисунка
\setlength\fboxrule{1pt} % Толщина линий рамки \fbox{}
\usepackage{wrapfig} % Обтекание рисунков и таблиц текстом

\usepackage{arcs}

\usepackage{geometry} % Простой способ задавать поля
\geometry{top=20mm}
\geometry{bottom=20mm}
\geometry{left=15mm}
\geometry{right=15mm}

\usepackage{euscript}
\usepackage{wasysym}

\usepackage{xcolor}
\usepackage{hyperref}

\usepackage{changepage}

% Цвета для гиперссылок
\definecolor{linkcolor}{HTML}{799B03} % цвет ссылок
\definecolor{urlcolor}{HTML}{799B03} % цвет гиперссылок

\hypersetup{pdfstartview=FitH,  linkcolor=blue,urlcolor=red, colorlinks=true}

% переименовываем  список литературы в "список используемой литературы"
\addto\captionsrussian{\def\refname{Список используемой литературы}}

%\usepackage{fancyhdr} % Колонтитулы
%\pagestyle{fancy}
%\renewcommand{\headrulewidth}{0.3mm}  % Толщина линейки, отчеркивающей верхний колонтитул
%\lfoot{}
%\rfoot{}
%\rhead{}
%\chead{}
%\lhead{}